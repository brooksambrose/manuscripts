\documentclass[]{article}
\usepackage{lmodern}
\usepackage{amssymb,amsmath}
\usepackage{ifxetex,ifluatex}
\usepackage{fixltx2e} % provides \textsubscript
\ifnum 0\ifxetex 1\fi\ifluatex 1\fi=0 % if pdftex
  \usepackage[T1]{fontenc}
  \usepackage[utf8]{inputenc}
\else % if luatex or xelatex
  \ifxetex
    \usepackage{mathspec}
  \else
    \usepackage{fontspec}
  \fi
  \defaultfontfeatures{Ligatures=TeX,Scale=MatchLowercase}
\fi
% use upquote if available, for straight quotes in verbatim environments
\IfFileExists{upquote.sty}{\usepackage{upquote}}{}
% use microtype if available
\IfFileExists{microtype.sty}{%
\usepackage{microtype}
\UseMicrotypeSet[protrusion]{basicmath} % disable protrusion for tt fonts
}{}
\usepackage[unicode=true]{hyperref}
\hypersetup{
            pdfborder={0 0 0},
            breaklinks=true}
\urlstyle{same}  % don't use monospace font for urls
\IfFileExists{parskip.sty}{%
\usepackage{parskip}
}{% else
\setlength{\parindent}{0pt}
\setlength{\parskip}{6pt plus 2pt minus 1pt}
}
\setlength{\emergencystretch}{3em}  % prevent overfull lines
\providecommand{\tightlist}{%
  \setlength{\itemsep}{0pt}\setlength{\parskip}{0pt}}
\setcounter{secnumdepth}{0}
% Redefines (sub)paragraphs to behave more like sections
\ifx\paragraph\undefined\else
\let\oldparagraph\paragraph
\renewcommand{\paragraph}[1]{\oldparagraph{#1}\mbox{}}
\fi
\ifx\subparagraph\undefined\else
\let\oldsubparagraph\subparagraph
\renewcommand{\subparagraph}[1]{\oldsubparagraph{#1}\mbox{}}
\fi

\date{}

\begin{document}

\section{R Notebook}\label{r-notebook}

\section{1. Introduction}\label{introduction}

This collection of studies seeks to accomplish a simplistic goal. I wish
to enumerate schools of thought in the historical record of U.S. social
science scholarship. While this is little more than a census-taking
exercise, what will become interesting about it is what we must learn
when taking operationalization seriously. By attempting to resolve
difficulties that arise between ontology (what we assume is there) and
observation (what we see when we actually look), I am forced to update
my sociological imagination. While the studies below are empirical their
purpose is theoretical in the imaginary sense. I do not seek an unbiased
view of the world, but merely to move in an unbiased direction.

In Chapter 2 I begin by achieving clarity on what I think a school of
thought is ontologically, a particular set of patterns among
personalities, cultures, and social structures. From the perspective of
this orienting theory in Chapter 3 I take a critical look at the work of
three theorists of intellectual development, Jen Lena, Thomas Kuhn,
Harry Collins. In Chapter 4 I motivate the selection of the U.S. social
sciences as a case of intellectual development. In Chapter 5 I introduce
the source material for the studies that follow. In Chapter 6 I execute
five brief empirical studies that test different aspects of my
expectations about intellectual communities and their development. In
Chapters 7 and 8 I summarize findings and discuss limitations of the
study. In Chapter 9 I conclude by updating the theory building of
Chapters 2 and 3 with the results of the studies.

\section{2. Theory}\label{theory}

\subsection{A. Cultural}\label{a.-cultural}

\subsection{B. Personal}\label{b.-personal}

\subsection{C. Social}\label{c.-social}

\subsection{D. Facilitory}\label{d.-facilitory}

\subsection{E. Developmental}\label{e.-developmental}

\section{3. Critical Review}\label{critical-review}

\subsection{Lena}\label{lena}

\subsection{Kuhn}\label{kuhn}

\subsection{Collins}\label{collins}

\subsection{Abbott}\label{abbott}

\section{4. Historical Case: U.S. Social Sciences,
1900-1942}\label{historical-case-u.s.-social-sciences-1900-1942}

\section{5. Data}\label{data}

\subsection{Google Books}\label{google-books}

\subsection{Thompson Reuters Web of
Knowledge}\label{thompson-reuters-web-of-knowledge}

\subsection{JSTOR Data for Research}\label{jstor-data-for-research}

\subsection{Sample Selection}\label{sample-selection}

\section{6. Studies}\label{studies}

\subsection{Study 1: Disciplinary
Prefixes}\label{study-1-disciplinary-prefixes}

\begin{itemize}
\tightlist
\item
  Ordered Diffusion of Disciplinary Bases: A Time Series Analysis
\end{itemize}

\subsection{Study 2: Settlement of Citation
Landscape}\label{study-2-settlement-of-citation-landscape}

\begin{itemize}
\tightlist
\item
  Poisson Permutation Test to Locate Transition from Extensive to
  Intensive Development
\end{itemize}

\subsection{Study 3: Subdiscipline
Formation}\label{study-3-subdiscipline-formation}

\begin{itemize}
\tightlist
\item
  k-Clique Percolation Clustering of Co-reference Network
\end{itemize}

\subsection{Study 4: Survival of
Knowledge}\label{study-4-survival-of-knowledge}

\begin{itemize}
\tightlist
\item
  Cox Proportional Hazard Analysis of k-Cliques
\end{itemize}

\subsection{Study 5: Genre Signaling}\label{study-5-genre-signaling}

\begin{itemize}
\tightlist
\item
  Association between k-Clique and Topic Model Classifications
\end{itemize}

\section{7. Summary of Findings}\label{summary-of-findings}

\section{8. Limitations}\label{limitations}

\section{9. Conclusion}\label{conclusion}

\section{10. Appendices}\label{appendices}

\subsection{Codebase}\label{codebase}

\subsection{Reproducible Software
Container}\label{reproducible-software-container}

\end{document}
